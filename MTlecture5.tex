\classheader{Lecture 5}
\setcounter{lecture}{5}
\begin{center}
	\Large \bf  Monotone Classes
\end{center}
\vspace{0.25cm}

\begin{definition}
	Given $ \Omega $, define $ \mathcal{M}\left(a\right) \subseteq \mathcal{P}(\Omega) $ is a monotone class is 
	\begin{enumerate}
		\item $ {A_j} \in \mathcal{M},{A_j} \uparrow A\left( {{A_j} \subseteq {A_j},\bigcup\limits_{j \geqslant 1} {{A_j}}  = A} \right) \Rightarrow A \in \mathcal{M} $
		\item ${A_j} \in \mathcal{M},{A_j} \downarrow A\left( {{A_j} \supseteq {A_j},\bigcap\limits_{j \geqslant 1} {{A_j}}  = A} \right) \Rightarrow A \in \mathcal{M}$
	\end{enumerate}
   \label{def5.1}
\end{definition}
 
\begin{remark}
	\text{}
	\begin{enumerate}
		\item $ \mathcal{F} $ is $ \sigma $-filed($ \sigma $-algebra) $ \Rightarrow \mathcal{F}$ is a monotone class
		\item  ${\mathcal{M}_\alpha } \subseteq P\left( \Omega  \right),\;\left( {\alpha  \in I} \right)\;$ is monotone class, then $ \mathcal{M}= \bigcap\limits_{\alpha  \in I} {{\mathcal{M}_\alpha }} $ is a monotone class.
	\end{enumerate}
	\label{rmk5.1}
\end{remark}

\begin{notation}(Smallest monotone class contain $ c $)
	$ \mathcal{M}(c) $ is  a monotone class generated by $ c $ if 
	\begin{equation}
	c \subseteq \mathcal{M}(\Omega), \mathcal{M}\left( c \right) = \bigcap\limits_{\alpha  \in I} {{\mathcal{M}_\alpha }} 
	\label{eq5.1}
	\end{equation}
	\label{not5.1}
\end{notation}

\begin{definition}
	$ E \subseteq \mathcal{M}(a) $, the set $ \mathcal{G}(E) $ is defined as below %$ E \subseteq \mathcal{M}(a) $
	\begin{equation}
	\mathcal{G}\left( E \right) = \left\{ {F \in \mathcal{M}\left( a \right),E\backslash F,E \cap F,F\backslash E \in \mathcal{M}\left( a \right)} \right\}
	\label{eq5.2}
	\end{equation}
	\label{def5.2}
\end{definition}

\begin{lemma}
	\text{}
	\begin{enumerate}
		\item If $ E \in a \Rightarrow \mathcal{G}(E) \supseteq \mathcal{M}(a) $
		\item If $ E \in \mathcal{M}(a) \Rightarrow \mathcal{G}(E) \supseteq \mathcal{M}(a) $
	\end{enumerate}
\label{lmk5.1}
\end{lemma}

\begin{proof}
	\text{}
	\begin{enumerate}
		\item  $ E \in a $,  we want to show that 
		\begin{enumerate}
			\item $ \mathcal{G}(E) \supseteq a $
			
			Take $ H \in a \subseteq \mathcal{M}(a)$, then
			\begin{equation}
			\underbrace {E\backslash H}_{ \in \;a},\underbrace {E \cap H}_{ \in \;a},\underbrace {H\backslash E}_{ \in \;a} \in \mathcal{G}\left( a \right)
			\label{eq5.3}
			\end{equation}
			so $H \in \mathcal{G}\left( E \right)$, then $a \subseteq \mathcal{G}\left( E \right)$
			\item $ \mathcal{G}(E) $ is a monotone class
			
			Suppose that ${H_k} \uparrow H,\;{H_k} \in \mathcal{G}\left( E \right)$,
			\begin{equation}
			\because  E\backslash {H_k} \in \mathcal{M}\left( a \right),\;E\backslash {H_k} \to E\backslash H , \therefore  E\backslash H \in \mathcal{M}\left( a \right)
			\label{eq5.4}
			\end{equation}
			\begin{equation}
			\because E \cap {H_k} \in \mathcal{M}\left( a \right),\;E \cap {H_k} \to E \cap H,\therefore E \cap H \in {\mathcal{M}}\left( a \right)
			\label{eq5.5}
			\end{equation}
			\begin{equation}
			\because {H_k}\backslash E \in \mathcal{M}\left( a \right),\;{H_k}\backslash E \to H\backslash E,\therefore H\backslash E \in \mathcal{M}\left( a \right)
			\label{eq5.6}
			\end{equation}
			By Eq \ref{eq5.6}, $ H \in \mathcal{M}(a), $ and by the definition \ref{def5.2}, $ H \in \mathcal{G}(E). $ So $ \mathcal{G}(E) $ is a monotone class. We also get that
			$ \mathcal{G}(E)  \supseteq \mathcal{M}(a).$
		\end{enumerate}
		\item $ E \in \mathcal{M}(a) $, we want to show that 
		\begin{enumerate}
			\item $ \mathcal{G}(E) $ is a monotone class 
			
			$ E \in \mathcal{M}(a) $, suppose ${H_k} \in \mathcal{G}\left( E \right),{H_k} \uparrow H$
			\begin{equation}
			\because E\backslash {H_k} \in \mathcal{M}\left( a \right),E\backslash {H_k} \downarrow E\backslash H\;\;\therefore E\backslash H \in \mathcal{M}\left( a \right)
			\label{eq5.7}
			\end{equation}
			Similarity:
			\begin{equation}
			E \cap H \in \mathcal{M}\left( a \right)
			\label{eq5.8}
			\end{equation}
			\begin{equation}
			H\backslash E \in \mathcal{M}\left( a \right)
			\label{eq5.9}
			\end{equation}
			then we can get $H \in \mathcal{G}\left( E \right)$, so $ \mathcal{G}\left( E \right) $ is a monotone class.
			\item $ \mathcal{G}(E) \supseteq a $ 
			
		     We need to show $ H \in a \Rightarrow H \in \mathcal{G}(E) $.
		     
		     By Lemma \ref{rmk5.1}.1, we can get that
		     \begin{equation}
		     \mathcal{G}(H) \supseteq \mathcal{M}(a)
		     \label{eq5.10}
		     \end{equation}
		     $ \because E \in \mathcal{M}(a), \therefore E \in \mathcal{G}(H) $, by the Def \ref{def5.2}, $H\backslash E,H \cap E,E\backslash H \in \mathcal{M}\left( a \right)$, so we can get  $ a \in \mathcal{G}\left( E \right) $
		\end{enumerate}
	\end{enumerate}
\end{proof}

\begin{theorem}
	$ a $ is a algebra, $ a \subseteq \mathcal{P}(\Omega) $.
	$ \mathcal{F}(a) $ is a $ \sigma $-algebra generated by $ a $, $ \mathcal{M}(a)  $ is a monotone class generated by $ a $, then
	\begin{equation}
	\mathcal{F}(a) =  \mathcal{M}(a)
	\label{eq5.11}
	\end{equation}
	\label{thm5.1}
\end{theorem}

\begin{proof}
	By remark \ref{rmk5.1}, $ \mathcal{F}(a) $ is a monotone class, by Notation \ref{not5.1} $ \mathcal{F}(a) \supseteq a $ and $ \mathcal{F}(a) \supseteq \mathcal{M}(a)$. 
	
	So we have to show that
	\begin{equation}
	\mathcal{F}(a) \subseteq \mathcal{M}(a)
	\label{eq5.12}
	\end{equation} 
	
	We will show that
	\begin{enumerate}
		\item $ \mathcal{M}(a) $ is a algebra
		\begin{enumerate}
			\item $\Omega  \in \mathcal{M}\left( a \right)$ by $ \Omega \subseteq a $
			\item $ E \in \mathcal{M}(a) \Rightarrow E^{c} \in \mathcal{M}(a) $
			
			By  Lemma \ref{lmk5.1}.1, let $ E = \Omega, $  then $ \mathcal{M}(a) \subseteq \mathcal{G}(\Omega) $. $ \because E \in \mathcal{M}(a) $, so $E \in \mathcal{G}(\Omega)$  . By  Definition \ref{def5.2}, $ \mathcal{G}\left( \Omega  \right) = \left\{ {E \in \mathcal{M}\left( a \right),{E^c},E,\emptyset  \in \mathcal{M}\left( a \right)} \right\} $
			\item $E,F \in \mathcal{M}\left( a \right) \Rightarrow E \cap F \in \mathcal{M}\left( a \right)$
			
			By  Lemma \ref{lmk5.1}.2, $\mathcal{G}\left( E \right) \supseteq \mathcal{M}\left( a \right)$, so $ F \in \mathcal{G}(E) $. 
			
			By Def \ref{def5.2} $F \in \mathcal{G}\left( E \right) = \left\{ {F \in \mathcal{M}\left( a \right),F\backslash E,F \cap E,E\backslash F \in \mathcal{M}\left( a \right)} \right\}$, so $ E \bigcap F \in \mathcal{M}(a) $
		\end{enumerate}
		\item $ \mathcal{M}(a) $ is a $ \sigma $-algebra i.e. ${A_j} \in \mathcal{M}\left( a \right),\;j \geqslant 1\; \Rightarrow \bigcup\limits_{j \geqslant 1} {{A_j}}  \in \mathcal{M}\left( a \right)$
		
		By $ \mathcal{M}(a) $ is a algebra, so $\bigcup\limits_{j = 1}^n {{A_j}}  \in \mathcal{M}\left( a \right)$.
		
		$\bigcup\limits_{j = 1}^n {{A_j}}  \uparrow \bigcup\limits_{j \geqslant 1} {{A_j}} $ and $ \mathcal{M}(a) $is a monotone class, so $\bigcup\limits_{j \geqslant 1} {{A_j}}  \in \mathcal{M}\left( a \right)$.
	\end{enumerate}
So $ \mathcal{F}(a) \subseteq \mathcal{M}(a) $.

Above all,
\begin{equation}
\mathcal{F}(a) =  \mathcal{M}(a)
\label{eq5.13}
\end{equation}
\end{proof}