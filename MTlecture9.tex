\classheader{Lecture 9}
\setcounter{lecture}{9}
\begin{center}
	\Large \bf Approximation Theorems
\end{center}
\vspace{0.25cm}

Goal: ${\pi ^*}\left( A \right) < \infty ,A \in \mathcal{M}, F \in \mathcal{F}, \ where \ \mathcal{F} \ is \ \sigma-algebra, A \subseteq F,{\pi ^*}\left( A \right) = {\pi ^*}\left( F \right)$.

\begin{theorem}
	$ a \subseteq \mathcal{P}(\Omega),\ where \  a $ is an algebra, $ \mathcal{F} $ is a $ \sigma- $algebra generated by $ a $, $ \mathcal{F}(a)=\mathcal{F} $, we have $\mu :\mathcal{F} \to \overline {{\mathbb{R}}}_ + $,  where  $ \mu $  is a measure, and  $\mu {|_a} = v$,  $ A \subseteq \mathcal{F}, \mu(A) < \infty, \forall \epsilon >0  $, there
	\begin{equation}
	\exists \;E \in a,\;s.t.\;\;\mu \left( {E\backslash A} \right) + \mu \left( {A\backslash E} \right) < \varepsilon
	\label{eq9.1}
	\end{equation}
	\label{thm9.1}
\end{theorem}

\begin{proof}
	$  A \in \mathcal{F}, \mu(A) < \infty $,  by Thm \ref{CaratheodoryTheorem}, then 
	\begin{equation}
	\mu \left( A \right) = {\pi ^*}\left( A \right) = \mathop {\inf }\limits_{\left\{ {{A_j}} \right\} \supseteq A,{A_{j \in a}}} \;\sum {\nu \left( {{A_i}} \right)} 
	\label{eq9.2}
	\end{equation}
	but $ \mu $ here  is $ \pi $ in Thm \ref{CaratheodoryTheorem}.
	
	$ \forall \epsilon, \exists \left\{ {{A_i}} \right\}\ \;{A_i} \in a,\;A \subseteq  \cup {A_i}, \ \ s.t. $
	\begin{equation}
	{\pi ^*}\left( A \right) \leqslant \sum\limits_{j \geqslant 1} {\nu \left( {{A_i}} \right)}  \leqslant {\pi ^*}\left( A \right) + \varepsilon 
	\label{eq9.3}
	\end{equation}
	so 
	\begin{equation}
	\exists \ m_{0} , \ \ s.t. \sum\limits_{i \geqslant {m_0}} {\nu \left( {{A_i}} \right)}  \leqslant \varepsilon
	\label{eq9.4}
	\end{equation}
	
	Let $ E = \bigcup\limits_{i = 1}^{{m_0}} {{A_i}}  \in a $, then we need to proof the following:
	\begin{equation}
	{\pi ^*}\left( {E\backslash A} \right) \leqslant \varepsilon, \;\;\;{\pi ^*}\left( {A\backslash E} \right) \leqslant \varepsilon 
	\label{eq9.5}
	\end{equation}
	
	By Thm \ref{thm4.1}, ${\pi ^*}\left( A \right)$ is an out-measure, ${\pi ^*}\left( A \right)$ is monotone and by Tmm \ref{thm4.3}, ${\pi ^*}\left( A \right)$ is  $ \sigma $-additive. 
	
	\begin{equation}
	\begin{split}
	\therefore {\pi ^*}\left( {E\backslash A} \right) & = {\pi ^*}\left( {\bigcup\limits_{i = 1}^{{n_0}} {{A_i}} \backslash A} \right)\\
													  & \leqslant {\pi ^*}\left( {\bigcup\limits_{i \geqslant 1} {{A_i}\backslash A} } \right)\\
													  &  = {\pi ^*}\left( {\bigcup\limits_{i \geqslant 1} {{A_i}} } \right) - {\pi ^*}\left( A \right)\;\;\;by\;{\pi ^*}\left( A \right) = \mu \left( A \right) < \infty \\
													  & \leqslant \sum\limits_{i \geqslant 1} {{\pi ^*}\left( {{A_i}} \right)}  - {\pi ^*}\left( A \right)\\
													  & =  \sum\limits_{i \geqslant 1} {\nu \left( {{A_i}} \right)}  - {\pi ^*}\left( A \right)\;by\;{\pi ^*}{|_{\mathcal{F}}} = \mu ,\;\mu {|_a} = v,\;{A_i} \in a\therefore {\pi ^*}\left( {{A_i}} \right) = \nu \left( {{A_i}} \right)\\
													  & \le  \varepsilon
	\end{split}
	\label{eq9.6}
	\end{equation}
	
	On the other hand,
	\begin{equation}
	\small 
	{\pi ^*}\left( {A\backslash E} \right) = {\pi ^*}\left( {A\backslash \bigcup\limits_{i = 1}^{{n_0}} {{A_i}} } \right)\leqslant {\pi ^*}\left( {\bigcup\limits_{i \geqslant 1} {{A_i}} \backslash \bigcup\limits_{j = 1}^{{n_0}} {{A_j}} } \right)\leqslant {\pi ^*}\left( {\bigcup\limits_{j \geqslant {n_0} + 1}^{{n_0}} {{A_j}} } \right)  \leqslant \sum\limits_{j \geqslant {m_0}} {\left( {\bigcup\limits_{j \geqslant {n_0} + 1}^{{n_0}} {{A_j}} } \right)} \leqslant \varepsilon 								
	\label{eq9.7}
	\end{equation}
	
\end{proof}

\begin{remark}
	$ \Omega $ is $ \sigma-$finite$ (\mu) $ ( i.e. $\Omega  = \bigcup\limits_{i \geqslant 1} {{E_i}} \;where\;{E_i} \in a,\mu \left( {{E_i}} \right) < \infty $), $\overline \mu  :\overline {\mathcal{F}}  \to {{\mathbb{R}}_ + } \cup \left\{ { + \infty } \right\},\;A \in \overline {\mathcal{F}} ,\forall \varepsilon  > 0,\exists E \in a$, such that
	\begin{equation}
	\overline \mu  \left( {E\backslash A} \right) + \overline \mu  \left( {A\backslash E} \right) < \varepsilon .
	\label{eq9.8}
	\end{equation}
	\label{rmk9.1}
\end{remark}


$\Omega$ is topological space (open, closed sets), $\mathcal{B}$ is Borel $\sigma$-algebra set (the smallest $\sigma$ set which contains all open, closed sets in $\Omega$).

\begin{definition}[Regular Measure]
	$ \mu: \mathcal{F} \to  {\mathbb{R}_ + } \cup \left\{ \infty  \right\} $ where $\mathcal{B} \subseteq \mathcal{F}$, is a measure. Then $ \mu $ is a regular measure if: $ \forall A \in \mathcal{F}, \forall \epsilon >0 $, there $\exists F \subseteq A \subseteq G,$ where $ F \in \mathcal{B}$ closed, $ G \in \mathcal{B}$ open, such that:
	\begin{equation}
	\mu \left( {G\backslash F} \right) \leqslant \varepsilon 
	\label{eq9.9}
	\end{equation}
	\label{def9.1}
\end{definition}

\begin{remark}
	$ \mu < \infty$ is not necessary.
	\label{rmk9.2}
\end{remark}

\begin{remark}
	$ \mu \left( {G\backslash A} \right) \leqslant \varepsilon $   and  $ \mu \left( {A\backslash F} \right) \leqslant \varepsilon $. 
	\label{rmk9.3}
\end{remark}

\begin{remark}
	$\mathcal{B} \subseteq \mathcal{F},\;\mu \;is\;regular\; \Rightarrow \;\mathcal{F} \subseteq \overline {{\mathcal{B}_\mu }} $
	\label{rmk9.4}
\end{remark}

\begin{proof}
	$ 	A \in \mathcal{F}, n \ge 1, \; by \;  \mu $ is regular, then $ \exists F_{n}, G_{n} \in \mathcal{B}, F_{n} \subseteq \mathcal{B}, $ such that $\mu \left( {{F_n}\backslash {G_n}} \right) \leqslant \frac{1}{n}$. 
	
	Let's define $F = \bigcup\limits_{n \geqslant 1} {{F_n}}  \in \mathcal{B},\;G = \bigcap\limits_{n \geqslant 1} {{G_n}}  \in \mathcal{B}$, then $F \subseteq {F_n} \subseteq A \subseteq {G_n} \subseteq G,\;i.e.\;F \subseteq A \subseteq G$.
	By
	\begin{equation}
	\small 
	{G_n}\backslash \left( {\bigcup\limits_{k \geqslant 1} {{F_k}} } \right) = {G_n} \cap {\left( {\bigcup\limits_{k \geqslant 1} {{F_k}} } \right)^c} = {G_n} \cap \left( {\bigcap\limits_{k \geqslant 1} {F_k^c} } \right) = \bigcap\limits_{k \geqslant 1} {\left( {{G_n} \cap F_k^c} \right)}  = \bigcap\limits_{k \geqslant 1} {\left( {{G_n}\backslash {F_k}} \right)}  \subseteq {G_n}\backslash {F_n}
	\label{eq9.10}
	\end{equation}
	then 
	\begin{equation}
	\mu \left( {G\backslash F} \right) \leqslant \mu \left( {{G_n}\backslash \left( {\bigcup\limits_{k \geqslant 1} {{F_k}} } \right)} \right) \leqslant \mu \left( {{G_n}\backslash {F_n}} \right) \leqslant \frac{1}{n} \to 0
	\label{eq9.11}
	\end{equation}
	
	Finally, 
	
	\begin{equation}
	A = \underbrace F_{ \in \mathcal{B}} \cup \underbrace {\left( {A\backslash F} \right)}_{ \subseteq G\backslash F \in \mathcal{B}} \in \mathcal{B} \Rightarrow A \in \overline {\mathcal{B}} 
	\label{eq9.12}
	\end{equation}
\end{proof}

\begin{theorem}
	$\mathcal{L}$ is a $\sigma$-algebra generated by $ a(\mathcal{S}) $, where $ \mathcal{S} $ is a set which defined as in Lecture 7, i.e. $ \mathcal{S} = \left\{ {\emptyset ,\mathbb{R},\left( {a,b} \right],\left( {a,\infty } \right),\left( { - \infty ,b} \right]} \right\}.$ $\mu :\mathcal{L} \to {\mathbb{R}_ + } \cup \left\{ \infty  \right\}$, is Lebesgue measure, then $ \mu  $ is regular measure. (if $ A \in \mathcal{L}$, there $\exists F$closed, $ G $ open, $F \subseteq A \subseteq G$ such that $\mu \left( {G\backslash F} \right) \leqslant \varepsilon $).
	\label{thm9.2}
\end{theorem}

\begin{proof}
	\text{}
	\begin{enumerate}
		\item goal: $ A \in \mathcal{L}, \varepsilon >0 $, there exists G open, such that $A \subseteq G,\;\mu \left( {G\backslash A} \right) \leqslant \varepsilon $.
		
		Denote ${E_n} = \left[ { - n,n} \right]$, ${A_n} = A \cap {E_n}$, then $\mu \left( {{A_n}} \right) < \infty $. By the construction of Caratheodory Thm \ref{CaratheodoryTheorem}, there $\exists {\left\{ {{B_{n,k}}} \right\}_{k \geqslant 1}},{B_{n,k}} \in a,{A_n} \subseteq \bigcup\limits_{k \geqslant 1} {{B_{n,k}}} ,$ such that 
		\begin{equation}
		\mu \left( {{A_n}} \right) \leqslant \sum\limits_{k \geqslant 1} {\mu \left( {{B_{n,k}}} \right)}  \leqslant \mu \left( {{A_n}} \right) + \frac{\varepsilon }{{{2^n}}}
		\label{eq9.13}
		\end{equation} 
		By ${B_{n,k}} \in a,\;\therefore {B_{n,k}} = \sum\limits_{j = 1}^{{l_{n,k}}} {{I_{n,k,j}}}  \subseteq {G_{n,k}}$, where ${I_{n,k,j}} = \left( {{a_{n,k,j}},{b_{n,k,j}}} \right]$. 
		
		Then we denote ${c_{n,k,j}} = {b_{n,k,j}} + \underbrace {{\delta _{n,k,j}}}_{ > 0},{J_{n,k,j}} = \left( {{a_{n,k,j}},{c_{n,k,j}}} \right)$, then ${B_{n,k}} \subseteq {G_{n,k}} = \bigcup\limits_{j = 1}^{{l_{n,k}}} {{J_{n,k,j}}} $, then 
		\begin{equation}
	    \mu \left( {{G_{n,k}}} \right) \leqslant \sum\limits_{j = 1}^{{l_{n,k}}} {\mu \left( {{I_{n,k,j}}} \right) + {\delta _{n,k,j}}}  = \underbrace {\sum\limits_{j = 1}^{{l_{n,k}}} {\mu \left( {{I_{n,k,j}}} \right)} }_{\mu \left( {{B_{n,k}}} \right)} + \underbrace {\sum\limits_{j = 1}^{{l_{n,k}}} {{\delta _{n,k,j}}} }_{ \leqslant \frac{\varepsilon }{{{2^n}{2^k}}}}
	    \label{eq9.14}
		\end{equation}
		
		$ \because {B_{n,k}} \subseteq {G_{n,k,}}\;and\;{G_{n,k}}\;open\;set\; $ $\therefore \mu \left( {{G_{n,k}}} \right) \leqslant \mu \left( {{B_{n,k}}} \right) + \frac{\varepsilon }{{{2^n}{2^k}}}$. $\because {A_n} \subseteq \bigcup\limits_{k \geqslant 1} {{B_{n,k}}} ,{B_{n,k}} \subseteq {G_{n,k}}\therefore {A_n} \subseteq \bigcup\limits_{k \geqslant 1} {{G_{n,k}}}  = {G_n}$.
		
        On the other hand, 
        \begin{equation}
        \mu \left( {{G_n}} \right) \leqslant \sum\limits_{k \geqslant 1} {\mu \left( {{G_{n,k}}} \right)}  \leqslant \sum\limits_{k \geqslant 1} {\mu \left( {{B_{n,k}}} \right)}  + \frac{\varepsilon }{{{2^n}}} \leqslant \mu \left( {{A_n}} \right) + \frac{{2\varepsilon }}{{{2^n}}}
        \label{eq9.15}
        \end{equation}
        $\because {A_n} \subseteq {G_n}\;open,\;and\;\mu \left( {{G_n}} \right) \leqslant \mu \left( {{A_n}} \right) + \frac{{2\varepsilon }}{{{2^n}}}$.
        
        Then define $G = \bigcup\limits_{n \geqslant 1} {{G_n}} ,\;open\;and\;\;A = \bigcup\limits_{n \geqslant 1} {{A_n}} ,\;A \subseteq G$.
        
        \begin{equation}
        \begin{split}
		\because \bigcup\limits_{n \geqslant 1} {{G_n}} \backslash \bigcup\limits_{k \geqslant 1} {{A_k}}  & = \bigcup\limits_{n \geqslant 1} {{G_n}}  \cap {\left( {\bigcup\limits_{k \geqslant 1} {{A_k}} } \right)^c} = \bigcup\limits_{n \geqslant 1} {{G_n}}  \cap \left( {\bigcap\limits_{k \geqslant 1} {A_k^c} } \right)\\
						&   = \bigcap\limits_{k \geqslant 1} {\left( {\bigcup\limits_{n \geqslant 1} {{G_n}\bigcap {A_k^c} } } \right) \subseteq \left( {\bigcup\limits_{n \geqslant 1} {{G_n}\bigcap {A_n^c} } } \right)} =\bigcup\limits_{n \geqslant 1} {{G_n}\backslash {A_n}} 
        \end{split}
        \label{eq9.16}
        \end{equation}
        \begin{equation}
        \begin{split}
        \therefore \mu \left( {G\backslash A} \right) & = \mu \left( {\bigcup\limits_{n \geqslant 1} {{G_n}} \backslash \bigcup\limits_{k \geqslant 1} {{A_k}} } \right)\\
        											  & \leqslant \mu \left( {\bigcup\limits_{n \geqslant 1} {{G_n}} \backslash {A_n}} \right)\;\;\;by\;\;Eq. \;\ref{eq9.16}\\
        											  & \leqslant \sum\limits_{n \geqslant 1} {\mu \left( {{G_n}\backslash {A_n}} \right)} \\
        											  & = \sum\limits_{n \geqslant 1} {\left[ {\mu \left( {{G_n}} \right) - \mu \left( {{A_n}} \right)} \right]} \;\;\;\;by\;\mu \left( {{A_n}} \right) < \infty \\
        											  & \le 2 \varepsilon
        \end{split}
        \end{equation}
		\item goal: $ A \in \mathcal{L}, \varepsilon >0 $, there exists F closed , such that $F \subseteq A,\;\mu \left( {A\backslash F} \right) \leqslant \varepsilon $.
		
		By above 1, $\exists H,\;{A^c} \subseteq H,\;H\;open\;set,\;\mu \left( {H\backslash {A^c}} \right) \leqslant \varepsilon ,\;then\;F = {H^c} \subseteq A,\;F\;closed\;$.
		
		Finally,
		\begin{equation}
		\mu \left( {A\backslash F} \right) = \mu \left( {A \cap {F^c}} \right) = \mu \left( {A \cap H} \right) = \mu \left( {H \cap {{\left( {{A^c}} \right)}^c}} \right) = \mu \left( {H\backslash {A^c}} \right) \leqslant \varepsilon .
		\label{eq.9.18}
		\end{equation}
	\end{enumerate}
\end{proof}

\begin{remark}
	${\mathcal{F}_\sigma }$: countable union closed sets, ${\mathcal{G}_\sigma }$: countable injection open sets. $ \forall A \in \mathcal{L} $ there $\exists R \in {\mathcal{F}_\sigma }$ and $ S \in {\mathcal{G}_\sigma } $, such that
	\begin{equation}
	R \subseteq A \subseteq S, \quad \mu \left( {S\backslash R} \right) = 0.
	\label{eq.9.19}
	\end{equation}
\end{remark}

 
 
