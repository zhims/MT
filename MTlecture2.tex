\classheader{Lecture 2}
\setcounter{lecture}{2}

\begin{center}
	 \bf Classes of Subsets (Semi-algebras, Algebras and Sigma-algebras) and Set Functions
\end{center}

\vspace{0.25cm}
 
\begin{definition}
	$ \mathcal{S} \subseteq \mathcal{P} \left( \Omega  \right) $, $ \mathcal{S} $ is semi-algebra if:
	\begin{enumerate}
		\item $ \Omega \subseteq \mathcal{S} $
		\item $ A,B \in \mathcal{S} \Rightarrow A\bigcap B \in \mathcal{S} $
		\item $ \forall A \in \mathcal{S} \Rightarrow {A^c} = \sum\limits_{i = 1}^n {{E_j}} ,\;\;\exists {E_1}, \cdots ,{E_n} \in \mathcal{S} $, $ E_{i},E_{j} \left(i \ne j\right) $ disjoint sets, $ n $ is finite number
	\end{enumerate}
	\label{def2.1}
\end{definition}

\begin{example}
	$\Omega  = \mathbb{R},\;\mathcal{S} = \left\{ {\mathbb{R},\left\{ {\left( {a,b} \right),a < b,a,b \in \mathbb{R}} \right\},\left\{ {\left( { - \infty ,b} \right],b \in \mathbb{R}} \right\},\left\{ {\left( {a,\infty } \right),a \in \mathbb{R}} \right\},\emptyset } \right\}$, ${\left( {a,b} \right]^c} = \left( { - \infty ,a} \right] \cup \left[ {b, + \infty } \right)$
\end{example}

\begin{example}
	\small  
	$\Omega  = {{\mathbb R}^2}$
	
	${\mathcal S} = {\text{ }}\left\{ {{{\mathbb R}^2}} \right.,\left\{ {\left( {{a_1},{b_1}} \right) \times \left( {{a_2},{b_2}} \right),\;{a_i} < {b_i},{a_i},{b_i} \in {\mathbb R},\left\{ {\left( { - \infty ,{b_1}} \right] \times \left( { - \infty ,{b_2}} \right],{b_i} \in {\mathbb R}} \right\},\left\{ {\left( {{a_1},\infty } \right) \times \left( {{a_2},\infty } \right),{a_i} \in {\mathbb R}} \right\},\emptyset } \right\}$
\end{example}

\begin{definition}
	$ a = \mathcal{P}\left(\Omega\right) $ is an algebra:
	\begin{enumerate}
		\item $ \Omega  \in a$ 
		\item $ A,B\in a \Rightarrow A\bigcap B \in a  $
		\item $ A \in a \Rightarrow A^{c} \in a  $
	\end{enumerate}
\end{definition}

\begin{remark}
	$ a $ algebra $ \Rightarrow $ $ a $ semi-algebra
\end{remark}

\begin{definition}
	$ \sigma$-algebra $ \mathcal{S} \subseteq \mathcal{P}\left(\Omega\right) $:
	\begin{enumerate}
		\item $ \Omega \subseteq \mathcal{S} $
		\item $ A_{j} \in \mathcal{S}, j\le 1 \Rightarrow \bigcap\limits_{j \geqslant 1} {{A_j}}  \in \mathcal{S} $
		\item $ A \in \mathcal{S} \Rightarrow {A^c} \in \mathcal{S}$
	\end{enumerate}
\end{definition}

\begin{remark}
	$ \Omega, a_{\alpha} \subseteq \mathcal{P}\left(\Omega\right) $, $ a_{\alpha} $ algebra, $ \alpha \in I $ $ \Rightarrow a = \bigcap\limits_{\alpha  \in I} {{a_\alpha }} $ is an algebra.
\end{remark}

\begin{proof}
	check the followings
	\begin{enumerate}
		\item $ \Omega \in a  $
		\item $ A,B \in a \Rightarrow A \bigcap B \in a $
		\item $ A \in a \Rightarrow A^{c} \in a $
	\end{enumerate}
\end{proof}

\begin{remark}
	$ \Omega, a_{\alpha} \subseteq \mathcal{P}\left(\Omega\right), \alpha \in I, a_{\alpha} $, $ \sigma $-algebra $ \Rightarrow a = \bigcap\limits_{\alpha  \in I} {{a_\alpha }} $  is a $ \sigma $-algebra
\end{remark}

\begin{proof}
	check the followings
	\begin{enumerate}
		\item $ \Omega \in a  $
		\item $ A_{j}, j \ge 1 \in a \Rightarrow \bigcap\limits_{j \geqslant 1} {{A_j}}  \in a$
		\item $ A \in a \Rightarrow A^{c} \in a $
\end{enumerate}
\end{proof}

\begin{definition}[ minimal algebra generated by $ c $]
	$ \Omega, c \subseteq \mathcal{P}\left(\Omega\right) $, $ a \left(c\right) $ is an algebra generated by $ c $, and $ a = a\left(c\right) $:
	\begin{enumerate}
		\item $ c \subseteq a $ 
		\item $ \forall  \mathcal{B} $ is algebra, $ \mathcal{B} \subseteq \mathcal{P}\left(\Omega\right) $:
		\begin{equation}
		 c \subseteq \mathcal{B} \Rightarrow a \subseteq \mathcal{B} 
		\end{equation}
	\end{enumerate}
\end{definition}

\begin{remark}
	$ a\left(c\right) $ exits, and $a = a\left( c \right) = \bigcap\limits_\alpha  {{a_\alpha }} ,\;\forall \alpha ,\;c \subseteq {a_\alpha },\;{a_\alpha }$ is an algebra.
\end{remark}

\begin{definition}[ minimal $ \sigma $-algebra generated by $ c $]
	$ \Omega, c \subseteq \mathcal{P}\left(\Omega\right) $, $ a \left(c\right) $ is a $ \sigma $-algebra generated by $ c $, and $ a = a\left(c\right) $:
	\begin{enumerate}
		\item $ c \subseteq a $ 
		\item $ \forall  \mathcal{B} $ is $ \sigma $-algebra, $ \mathcal{B} \subseteq \mathcal{P}\left(\Omega\right) $:
		\begin{equation}
		c \subseteq \mathcal{B} \Rightarrow a \subseteq \mathcal{B} 
		\end{equation}
	\end{enumerate}
\label{def2.5}
\end{definition}

\begin{remark}
	$ a\left(c\right) $ exits, and $a = a\left( c \right) = \bigcap\limits_\alpha  {{a_\alpha }} ,\;\forall \alpha ,\;c \subseteq {a_\alpha },\;{a_\alpha }$ is an $ \sigma $-algebra.
	\label{rmk2.5}
\end{remark}

\begin{lemma}
	$ \Omega, f $ semi-algebra $ f \subseteq \mathcal{P}\left(\Omega\right) $, $ a\left(f\right) $ algebra generated by $ f $ then 
	\begin{equation}
	A \in a\left(f\right) \Leftrightarrow \exists {E_j} \in f,1 \leqslant j \leqslant n,\;A = \sum\limits_{j = 1}^n {{E_j}} 
	\end{equation}
	\label{lma2.1}
\end{lemma}

\begin{proof}
	\text{}
	\begin{enumerate}
		\item $ \Leftarrow $ 
		
		$A = \sum\limits_{j = 1}^n {{E_j}} , \ {E_j} \in f \in a \left(f\right)$
		
		By definition \ref{def2.1} and remark \ref{rmk2.6} $ \Rightarrow A \in a \left(f\right) $
		\item $ \Rightarrow $
		
		$ A \in a\left(f\right) \Rightarrow A = \sum\limits_{j = 1}^n {{E_j}} ,{E_j} \in f $
		
		Then by remark \ref{rmk2.7}, it will be  proved easily.
	\end{enumerate}
\end{proof}

\begin{remark}
	$ E,J \in a, E \bigcup F \in a, E \bigcup F= \left(E^{c} \bigcap F^{c}\right)^{c} $
	\label{rmk2.6}
\end{remark}

\begin{remark}
	$ \mathcal{B} =\left\{ {\sum\limits_{j = 1}^n {{F_j},\;{F_j} \in f} } \right\},\;\mathcal{B} \subseteq \mathcal{P}\left( \Omega  \right) $ then
	\begin{enumerate}
		\item $ \mathcal{B} $ algebra
		\item $ \mathcal{B} \supseteq f $
		\item $ \mathcal{B} \supseteq a\left(f\right) $
	\end{enumerate} 
	\label{rmk2.7}
\end{remark}

\begin{proof}
	We only prove that  $ \mathcal{B} $ algebra, then check the following
	\begin{enumerate}
		\item $ \Omega \in \mathcal{B} $
		\item $A,B \in \mathcal{B} \Rightarrow A \cap B \in \mathcal{B}$
		
		$ \because A,B \in \mathcal{B}, $ $ \therefore A = \sum\limits_{j = 1}^n {{E_j}} ,\;{E_j} \in f,\;B = \sum\limits_{k = 1}^m {{F_k}} ,\;{F_k} \in f$, then 
		\begin{equation}
		\begin{split}
		A \cap B & = \left( {\sum\limits_{j = 1}^n {{E_j}} } \right) \cap \left( {\sum\limits_{k = 1}^m {{F_k}} } \right)\\
				 & = \sum\limits_{j = 1}^n {\sum\limits_{k = 1}^m {\underbrace {\left( {{E_j} \cap {F_k}} \right)}_{ \in f}} } \\
				 & \in  \mathcal{B}
		\end{split}
		\end{equation}
		
		\item $ A \in \mathcal{B} \Rightarrow A^{c} \in \mathcal{B} $
		
		$A = \sum\limits_{j = 1}^n {{E_j}} ,\;{E_j} \in f$ 
		
		By definition \ref{def2.1}:
		\begin{equation}
		\begin{split}
		E_1^c  & = \sum\limits_{{k_1} = 1}^{{l_1}} {{F_{1,{k_1}}}} ,\;{F_{1,j}} \in f \\
		\cdots & =  \cdots \\
		E_i^c & = \sum\limits_{{k_i} = 1}^{{l_i}} {{F_{i,{k_i}}}} ,\;{F_{i,j}} \in f
		\end{split}
		\end{equation}
		
		Then, we get that 
		\begin{equation}
		\begin{split}
		{A^c} & = \left( {\sum\limits_{{k_1} = 1}^{{l_1}} {{F_{1,{k_1}}}} } \right) \cap \left( {\sum\limits_{{k_2} = 1}^{{l_2}} {{F_{2,{k_2}}}} } \right) \cap  \cdots  \cap \left( {\sum\limits_{{k_n} = 1}^{{l_n}} {{F_{n,{k_n}}}} } \right)\\
		      &  = \sum\limits_{{k_1} = 1}^{{l_1}} {\sum\limits_{{k_2} = 1}^{{l_2}} { \cdots \sum\limits_{{k_n} = 1}^{{l_n}} {\left( {{F_{1,{k_1}}} \cap {F_{2,{k_2}}} \cap {F_{n,{k_n}}}} \right)} } } \\
		      & \in \mathcal{B}
		\end{split}
		\end{equation}
	\end{enumerate}
\end{proof}

\begin{definition}
	$ c \subseteq\mathcal{P}\left(\Omega\right), \emptyset \in c $, $\mu :\;c \to {\mathbb{R}_ + } \cup \left\{ { + \infty } \right\}$. $ \mu $ is additive if 
	\begin{enumerate}
		\item $\mu \left( \emptyset  \right) = 0$
		\item ${E_1},{E_2},...,{E_n} \in c,\;E = \sum\limits_{j = 1}^n {{E_j}}  \in c \Rightarrow \mu \left( E \right) = \sum\limits_{j = 1}^n {\mu \left( {{E_k}} \right)} $
	\end{enumerate}
	\label{def2.6}
\end{definition}

\begin{remark}
	\begin{equation}
	\exists A \in c,\;\mu \left( A \right) < \infty ,\;A = A \cup \emptyset ,\;\mu \left( A \right) = \mu \left( A \right) + \mu \left( \emptyset  \right) \Rightarrow \mu \left( \emptyset  \right) = 0
	\label{eq2.7}
	\end{equation}
	\label{rmk2.8}
\end{remark}

\begin{remark}
	$ c,\  \mu: c \to \mathbb{R}_{+} \bigcup {+ \infty}, \  E \subseteq F, \ F\backslash E \in c,\;E,F \in c $
	\begin{equation}
	F = E \cup \left( {F\backslash E} \right),\;\mu \left( F \right) = \mu \left( E \right) + \left( {F\backslash E} \right)
	\label{eq2.8}
	\end{equation}
	\begin{enumerate}
		\item $\mu \left( E \right) =  + \infty $, $\mu \left( F \right) =  + \infty $
		\item $\mu \left( E \right) <  + \infty $, $\mu \left( {F\backslash E} \right) = \mu \left( F \right) - \mu \left( E \right)$
	\end{enumerate}
	so, 
	\begin{equation}
	\mu \left( E \right) \leqslant \mu \left( F \right)
	\label{eq2.9}
	\end{equation}
	\label{rmk2.9}
\end{remark}

\begin{example}
	Discrete measure:
	$ \Omega $, $ c \subseteq  \mathcal{P}\left(\Omega\right) $, $\left\{ {{x_j},\;j \geqslant 1} \right\},\;{x_j} \in \Omega ,\;\left\{ {{p_j},\;j \geqslant 1} \right\},\;{p_j},  \geqslant 0$, $ A \in c, $ define that
	\begin{equation}
	\mu \left( A \right) = \sum\limits_{j \geqslant 1} {{p_j}} 1\left\{ {{x_j} \in A} \right\}
	\end{equation}
	then $ \mu $ is additive
	\label{eg2.3}
\end{example}

\begin{definition}
	$ c \in \mathcal{P}\left(\Omega\right), \emptyset \in c, $ $ \mu: \  c \to \mathbb{R}_{+} \bigcup {+ \infty }$, $ \mu $ is $ \sigma $-additive if  
	\begin{enumerate}
		\item $\mu \left( \emptyset  \right) = 0$
		\item ${E_j}\in c, \ j \ne k, E_{j} \bigcap E_{k} = \emptyset,\ \;E = \sum\limits_{j \ge  1} {{E_j}}  \in c \Rightarrow \mu \left( E \right) = \sum\limits_{j \ge 1} {\mu \left( {{E_j}} \right)} $
	\end{enumerate}
\end{definition}

\begin{example}
	$\Omega  = \left( {0,1} \right),\;c = \left\{ {\left( {a,b} \right],\;0 \leqslant a < b < 1} \right\},$ $\mu :\;c \to {\mathbb{R}_ + } \cup \left\{ { + \infty } \right\}$,  define that
	\begin{equation}
	\mu \left( {a,b} \right] = \left\{ {\begin{matrix}
		{ + \infty }  \\ 
		{b - a}  \\ 	
		\end{matrix} } \right.\;\;\begin{matrix}
	{a = 0}  \\ 
	{a > 0}  \\ 
	\end{matrix} 
	\end{equation}
	$\left( {a,b} \right] = \sum\limits_{j = 1}^n {\left( {{a_j},{b_j}} \right)} $,  we can get that  $ \mu $ is NOT $ \sigma $-additive.
	
	If  $ {x_1} = \frac{1}{2},{x_j} > {x_{j + 1}},{x_j} \downarrow  \to 0 $, then 
	\begin{equation}
	\frac{1}{2} = \left( {0,\frac{1}{2}} \right] = \sum\limits_{j \geqslant 1} {\left( {{x_{j + 1}},{x_j}} \right]}  =  + \infty 
	\end{equation}
	it is impossible.
\end{example}

\begin{definition}
	Any non-negative set function $\mu :C \to {{\mathbb{R}}_ + } \cup \left\{ { + \infty } \right\}$ which is $ \sigma-additive $ is called a measure on $ C $.
	\label{def2.8}
\end{definition}

