\classheader{Lecture 4}
\setcounter{lecture}{4}

\begin{center}
	\Large \bf Caratheodory Theorem
\end{center}

\vspace{0.25cm}

Intuition:
\begin{equation}
\begin{matrix}
{\sigma-add} & {\mu :f \to {\mathbb{R}_ + } \cup \left\{ { + \infty } \right\}} & {f \subseteq \mathcal{P}\left( \Omega  \right),f\;is\;semialgebra}  \\ 
\downarrow  &  \downarrow  & {}  \\ 
{\sigma-add} & {\nu :a\left( f \right) \to {\mathbb{R}_ + } \cup \left\{ { + \infty } \right\}} & {a\left( f \right)\;algebra \; generated \; by \; f}  \\ 
\downarrow  &  \downarrow  & {}  \\ 
{\sigma-add} & {\pi :\mathcal{F}\left( a \right) \to {\mathbb{R}_ + } \cup \left\{ { + \infty } \right\}} & {\mathcal{F}\left( a \right)\; is \; \sigma  - algebra\;generated\;by\;algebra}  \\ 
\end{matrix}
\label{eq4.1} 
\end{equation}

The big picture of the proof to the  Caratheodory Theorem:
\begin{enumerate}
	\item  Define the $ {\pi ^*} $ outer measure:
	\begin{equation}
	{\pi ^*} = \mathop {\inf }\limits_{\left\{ {{E_i}} \right\}} \sum\limits_{i \geqslant 1} {\nu \left( {{E_i}} \right)} 
	\label{eq4.2} 
	\end{equation}
	\item  $ \mathcal{M} $ $ \sigma $-algebra, $ \mathcal{M} \supseteq \mathcal{F}\left(a\right) $
	\item 
	\begin{equation}
	{\pi ^*}:\mathcal{M} \to {\mathbb{R}_ + } \cup \left\{ { + \infty } \right\}
	\label{eq4.3} 
	\end{equation}
	is $ \sigma $-additive, and 
	\begin{equation}
	{\pi ^*}{|_a} = \nu
	\label{eq4.4} 
	\end{equation}
	\item (uniqueness) $ \mu_{1},\mu_{2}: \mathcal{F}\left(a\right) \to \mathbb{R}_{+}\bigcup \{+\infty\} $, $ \Omega $ is $ \sigma $-finite($ \mu_{1} $), if $  E_{j} \uparrow \Omega $, ${\mu _1}\left( {{E_j}} \right) < \infty ,\forall j,\;{E_j} \in a$ and ${\mu _1}{|_a} = {\mu _2}{|_a}$ then implies that
	\begin{equation}
	{\mu _1} = {\mu _2}
	\label{eq4.5} 
	\end{equation}
\end{enumerate}
 
 \begin{equation}
 {\pi ^*}:\;\mathcal{P}\left( \Omega  \right) \to {\mathbb{R}_ + } \cup \left\{ { + \infty } \right\}
 \label{eq4.6}
 \end{equation}
 
We will prove $ {\pi ^*} $ is an outer measure.

And we will construct a family of subsets $ \mathcal{M} $
\begin{equation}
\mathcal{M} \subseteq  \mathcal{P}\left(\Omega\right)
\label{eq4.7}
\end{equation}

we will also prove $ \mathcal{M} $ satisfies the following:
\begin{enumerate}
	\item $ \mathcal{M} $ is a $ \sigma- $algebra
	\item $ \mathcal{M}  \supseteq a$
	\item ${\pi ^*}{|_\mathcal{M}}$ $ \sigma- $additive
	\item ${\pi ^*}{|_a} = \nu $
\end{enumerate}

Next, we will define $ {\pi ^*} $ and $ \mathcal{M} $ .
\newpage 
 
 {\large Step 1}

\begin{definition}[$ {\pi ^*} $]
	$ {\pi ^*}:\;\mathcal{P}\left( \Omega  \right) \to {\mathbb{R}_ + } \cup \left\{ { + \infty } \right\} $, 
	$ A \in \Omega, $ $\left\{ {{E_i},i \geqslant 1} \right\},{E_i} \in a,A \subseteq \bigcup {{E_i}} $, $\left\{ {{E_i}} \right\}$ is a covering of A, then we define that
	\begin{equation}
	{\pi ^*} = \mathop {\inf }\limits_{\left\{ {{E_i}} \right\},A} \ \sum\limits_{i \geqslant 1} {\nu \left( {{E_i}} \right)} 
	\label{eq4.8}
	\end{equation}
	where $\nu :\;a\left( f \right) \to {\mathbb{R}_ + } \cup \left\{ { + \infty } \right\},\; is \; \sigma $-additive. 
	\label{def4.1}
\end{definition}



\begin{definition}[Outer measure]
	$\mu :\;c \to {\mathbb{R}_ + } \cup \left\{ { + \infty } \right\},c \subseteq P\left( \Omega  \right),\emptyset  \in c$, $ \mu  $ is a outer measure if
	\begin{enumerate}
		\item $\mu \left( \emptyset  \right) = 0$
		\item (monotone) $E \subseteq F,\;E,F \in c\; \Rightarrow \mu \left( E \right) \leqslant \mu \left( F \right)$
		\item (subadditive) $E,{E_i} \in c,\;E \subseteq \bigcup\limits_i {{E_i}}  \Rightarrow \mu \left( E \right) \leqslant \sum\limits_i {\mu \left( {{E_i}} \right)} $
	\end{enumerate}
	\label{def4.2}
\end{definition}

\begin{theorem}
	$ \pi^{*} $ in \ref{def4.1} is a outer measure.
	\label{thm4.1}
\end{theorem}

\begin{proof}
	We will check the conditions in  Def \ref{def4.2}. 
	\begin{enumerate}
		\item check ${\pi ^*}\left( \emptyset  \right) = 0$
		\begin{enumerate}
			\item ${E_i} = \emptyset ,\emptyset  \subseteq \bigcup\limits_{i \geqslant 1} {{E_i}} $ then
			\begin{equation}
			{\pi ^*}\left( \emptyset  \right) = \mathop {\inf }\limits_{\left\{ {{E_i}} \right\},\emptyset } \sum\limits_{i \geqslant 1} {\nu \left( {{E_i}} \right)}  \leqslant \sum\limits_{i \geqslant 1} {\nu \left( {{E_i}} \right)}  = 0
			\end{equation}
			\item ${E_i} \in a,\left\{ {{E_i}} \right\},\emptyset  \subseteq \bigcup\limits_{i \geqslant 1} {{E_i}} $, then
			\begin{equation}
			\sum\limits_{i \geqslant 1} {\nu \left( {{E_i}} \right)}  \geqslant 0 \Rightarrow {\pi ^*}\left( \emptyset  \right) \geqslant 0
			\end{equation}
		\end{enumerate}
		\item check $E \subseteq F,\;{\pi ^*}\left( E \right) \leqslant {\pi ^*}\left( F \right)$
		
		Let's take any covering of $ F $:$\left\{ {{E_i}} \right\},{E_i} \in a,F \subseteq \bigcup\limits_{i \geqslant 1} {{E_i}} $ is also a covering of $ E $, then
		\begin{equation}
		{\pi ^*}\left( E \right) = \mathop {\inf }\limits_{\left\{ {{E_i}} \right\},E} \sum\limits_{i \geqslant 1} {\nu \left( {{E_i}} \right)}  \leqslant {\pi ^*}\left( F \right) = \mathop {\inf }\limits_{\left\{ {{E_i}} \right\},F} \sum\limits_{i \geqslant 1} {\nu \left( {{E_i}} \right)} 
		\label{eq4.11}
		\end{equation}
		\item check $E \subseteq \bigcup\limits_{i \geqslant 1} {{E_i}} ,\;\;{\pi ^*}\left( E \right) \leqslant \sum\limits_{i \geqslant 1} {{\pi ^*}\left( {{E_i}} \right)} $
		\begin{enumerate}
			\item ${\pi ^*}\left( {{E_i}} \right) = \infty $ then 
			\begin{equation}
			{\pi ^*}\left( E \right) \leqslant \sum\limits_{i \geqslant 1} {{\pi ^*}\left( {{E_i}} \right)}
			\label{eq4.12}
			\end{equation}
			\item ${\pi ^*}\left( {{E_i}} \right) < \infty $, then
			\begin{equation}
			{\pi ^*}\left( {{E_i}} \right) = \mathop {\inf }\limits_{\left\{ {{H_{ik}}} \right\},\;{E_i}} \sum\limits_{k \geqslant 1} {\nu \left( {{H_{ik}}} \right)} 
			\label{eq4.13}
			\end{equation}
			then there $\exists \left\{ {{H_{ik}}} \right\} \in a,{E_i} \subseteq \bigcup\limits_{k \geqslant 1} {{H_{ik}}} $ such that
			\begin{equation}
			{\pi ^*}\left( {{E_i}} \right) = \mathop {\inf }\limits_{\left\{ {{H_{ik}}} \right\},\;{E_i}} \sum\limits_{k \geqslant 1} {\nu \left( {{H_{ik}}} \right)}  \leqslant \sum\limits_{k \geqslant 1} {\nu \left( {{H_{ik}}} \right)}  \leqslant {\pi ^*}\left( {{E_i}} \right) + \frac{\varepsilon }{{{2^i}}}
			\label{eq4.14}
			\end{equation}
			$\left\{ {{H_{ik}}} \right\}$ is a covering of E, then
			\begin{equation}
			\begin{split}
			{\pi ^*}\left( E \right) \leqslant \sum\limits_{i,k} {\nu \left( {{H_{ik}}} \right)}  \leqslant \sum\limits_{i \geqslant 1} {\left( {{\pi ^*}\left( {{E_i}} \right) + \frac{\varepsilon }{{{2^i}}}} \right)}  \leqslant \sum\limits_{i \geqslant 1} {{\pi ^*}\left( {{E_i}} \right) + \varepsilon } 
			\end{split}
			\label{eq4.15}
			\end{equation}
			so 
			\begin{equation}
			{\pi ^*}\left( E \right) \leqslant \sum\limits_{i \geqslant 1} {{\pi ^*}\left( {{E_i}} \right)} 
			\label{eq4.16}
			\end{equation}
		\end{enumerate}
	\end{enumerate}
\end{proof}

 {\large Step 2}
 
 \begin{definition}[Measurable set $ \mathcal{M} $]
 	A set called measurable set $ \mathcal{M} $ if $ A \in \mathcal{M} $  $ \forall E \in \Omega $, we have that
 	\begin{equation}
 	{\pi ^*}\left( E \right) = {\pi ^*}\left( {E\bigcap A } \right) + {\pi ^*}\left( {E\bigcap {{A^c}} } \right)
 	\label{eq4.17}
 	\end{equation}
 	\label{def4.3}
 \end{definition}

\begin{theorem}
	If $ \mathcal{M} $ definited as Def \ref{def4.3}, then 
	\begin{enumerate}
		\item $\mathcal{M} \supseteq a$
		\item $ \mathcal{M} $ is a $ \sigma- $algebra
	\end{enumerate}
     \label{thm4.2}
\end{theorem}

\begin{remark}
	\begin{equation}
	E \subseteq \left( {E \cap A} \right) \cup \left( {E \cap {A^c}} \right) \Rightarrow {\pi ^*}\left( E \right) \leqslant {\pi ^*}\left( {E \cap A} \right) + {\pi ^*}\left( {E \cap {A^c}} \right)
	\label{eq4.18}
	\end{equation}
	so we only to check $ \ge $ in Eq \ref{eq4.17}
	\label{rmk4.1}
\end{remark}

\begin{proof}
	${\pi ^*}$ is an outer measurable by Thm \ref{def4.1}, then by subadditive of outer measure.
\end{proof}

Now we proof Thm \ref{thm4.2}.
\begin{proof}
	\text{}
	\begin{enumerate}
		\item $ a \in \mathcal{M} $
		
		Suppose that $ A \in a, $  $ E \in \Omega $, we will show that
		\begin{equation}
		{\pi ^*}\left( E \right) \geqslant {\pi ^*}\left( {E \cap A} \right) + {\pi ^*}\left( {E \cap {A^c}} \right)
		\label{eq4.19}
		\end{equation}
		assume that ${\pi ^*}\left( E \right) < \infty$, given $\varepsilon ,\exists \left\{ {{E_i}} \right\},E,\;such\;\;that\;{E_i} \in a,E \subseteq \bigcup\limits_{i \geqslant 1} {{E_i}} $, then
		\begin{equation}
		{\pi ^*}\left( E \right) \leqslant \sum\limits_{i \geqslant 1} {\nu \left( {{E_i}} \right)}  \leqslant {\pi ^*}\left( E \right) + \varepsilon 
		\label{eq4.20}
		\end{equation}
		${E_i} \cap A \in a,E \cap A \subseteq \bigcup\limits_{i \geqslant 1} {\left( {{E_i}\bigcap A } \right)} $, so
		\begin{equation}
		\begin{split}
		{\pi ^*}\left( {{E} \cap A} \right) & \leqslant \sum\limits_{i \geqslant 1} {\nu \left( {{E_i}\bigcap A } \right)} \\
		{\pi ^*}\left( {{E} \cap {A^c}} \right) & \leqslant \sum\limits_{i \geqslant 1} {\nu \left( {{E_i}\bigcap {{A^c}} } \right)} 
		\end{split}
		\label{eq4.21}
		\end{equation}
		so 
		\begin{equation}
		{\pi ^*}\left( {E \cap A} \right) + {\pi ^*}\left( {E \cap {A^c}} \right) \leqslant \sum\limits_{i \geqslant 1} {\nu \left( {{E_i}\bigcap A } \right)}  + \sum\limits_{i \geqslant 1} {\nu \left( {{E_i}\bigcap {{A^c}} } \right)} \le \sum\limits_{i \geqslant 1} {\nu \left( {{E_i}} \right)}  \leqslant \pi^{*}\left( E \right) + \varepsilon 
		\label{eq4.22}
		\end{equation}
		\item $ \mathcal{M} $ is $ \sigma $-algebra.
		
		We need to show that
		\begin{enumerate}
			\item $ \Omega \in \mathcal{M} $
			
			It is clearly that:
			\begin{equation}
			{\pi ^*}\left( E \right) = {\pi ^*}\left( {E \cap \Omega } \right) + {\pi ^*}\left( {E \cap {\Omega ^c}} \right)
			\label{eq4.23}
			\end{equation}
			\item $ A \in \mathcal{M} \Rightarrow A^{c} \in \mathcal{M} $
			\begin{equation}
			\because {\pi ^*}\left( E \right) = {\pi ^*}\left( {E \cap A} \right) + {\pi ^*}\left( {E \cap {A^c}} \right)
			\label{eq4.24}
			\end{equation}
			\item ${A_i} \in \mathcal{M} \Rightarrow \bigcup\limits_{i \geqslant 1} {{A_i}}  \subseteq \mathcal{M}$
			
			Finite union is closed: $ A,B \in \mathcal{F} \Rightarrow A\bigcup {B \in M}. $ Let's take $E \subseteq \Omega $. We will proof that
			\begin{equation}
			{\pi ^*}\left( E \right) \geqslant {\pi ^*}\left( {E \cap \left( {A\bigcup B } \right)} \right) + {\pi ^*}\left( {E \cap {{\left( {A\bigcup B } \right)}^c}} \right)
			\label{eq4.25}
			\end{equation}
			$\because A \in \mathcal{M}$,
			\begin{equation}
			\therefore \;\;{\pi ^*}\left( E \right) = {\pi ^*}\left( {E\bigcap A } \right) + {\pi ^*}\left( {E\bigcap {{A^C}} } \right)
			\label{eq4.26}
			\end{equation}
			$\because B \in \mathcal{M}$
			\begin{equation}
			\begin{split}
			\therefore \;\;{\pi ^*}\left( {E\backslash A} \right) & = {\pi ^*}\left( {E\backslash A \cap B} \right) + {\pi ^*}\left( {E\backslash A \cap {B^c}} \right) \\
																& = {\pi ^*}\left( {E\backslash A \cap B} \right) + {\pi ^*}\left( {E\backslash \left( {A\bigcup B } \right)} \right)
			\end{split}
			\label{eq4.27}
			\end{equation}
			then 
			\begin{equation}
			{\pi ^*}\left( E \right) = {\pi ^*}\left( {E \cap A} \right) + {\pi ^*}\left( {E\backslash A \cap B} \right) + {\pi ^*}\left( {E\backslash \left( {A \cup B} \right)} \right)
			\label{eq4.28}
			\end{equation}
			We want to show
			\begin{equation}
			{\pi ^*}\left( {E \cap A} \right) + {\pi ^*}\left( {E\backslash A \cap B} \right) \geqslant {\pi ^*}\left( {E \cap \left( {A \cup B} \right)} \right)
			\label{eq4.29}
			\end{equation}
			By ${\pi ^*}$ is subadditive, we only to show that 
			\begin{equation}
			E \cap \left( {A \cup B} \right) \subseteq \left( {E \cap A} \right) \cup \left( {E\backslash A \cap B} \right)
			\label{eq4.30}
			\end{equation}
			this is because
			\begin{equation}
			E \cap \left( {A \cup B} \right) = \underbrace {\left\{ {\left[ {E \cap \left( {A \cup B} \right)} \right] \cap A} \right\}}_{ \subseteq E \cap A}\bigcup {\underbrace {\left\{ {\left[ {E \cap \left( {A \cup B} \right)} \right] \cap {A^c}} \right\}}_{ \subseteq \left( {E \cap {A^c}} \right) \cap B = \left( {E\backslash A} \right)\bigcap B }} 
			\label{eq4.31}
			\end{equation}
			Then Eq \ref{eq4.25} holds. So $ \mathcal{M} $ is closed by finite(countable) union.
			
			Now, we will show that countable infinite union is also closed. ${A_i} \in \mathcal{M}$, we want to show $A = \bigcup\limits_{j \geqslant 1} {{A_j}}  \in \mathcal{M}$, take $E \subseteq \Omega $,
			\begin{equation}
			{\pi ^*}\left( E \right) \geqslant {\pi ^*}\left( {E \cap A} \right) + {\pi ^*}\left( {E \cap {A^c}} \right)
			\label{eq4.32}
			\end{equation}
			by Eq. \ref{eq4.25}, $ \forall \ n $ we know that
			\begin{equation}
			\begin{split}
			{\pi ^*}\left( E \right) & = {\pi ^*}\left( {E \cap \left( {\bigcup\limits_{j = 1}^n {{A_j}} } \right)} \right) + {\pi ^*}\left( {E \cap \left( {\bigcup\limits_{j = 1}^n {A_{_j}^c} } \right)} \right)\\
			& \ge {\pi ^*}\left( {E \cap \left( {\bigcup\limits_{j = 1}^n {{A_j}} } \right)} \right) + {\pi ^*}\left( {E\backslash A} \right)
			\end{split}
			\label{eq4.33}
			\end{equation}
			$ \ge $ holds in Eq \ref{eq4.33} because $\left( {E\backslash A} \right) \subseteq \left( {E\backslash \left( {\bigcup\limits_{j = 1}^n {{A_j}} } \right)} \right)$.
			
			Now, we define
			\begin{equation}
			\begin{split}
			{F_1} & = {A_1}\\
			{F_2} & = {A_1}\backslash {A_2}\\
			{F_3} & = {A_1}\backslash \left( {{A_2} \cup {A_3}} \right)\\
			      & \quad \vdots \\
			{F_n} & = {A_1}\backslash \left( {{A_2} \cup  \cdots  \cup {A_{n - 1}}} \right)\\
				  & \quad  \vdots 
			\end{split}
			\label{eq4.34}
			\end{equation}
			It is clear that
			\begin{equation}
			\bigcup\limits_{i = 1}^n {{A_i}}  = \bigcup\limits_{j = 1}^n {{F_j}} 
			\label{eq4.35}
			\end{equation}
			where ${F_j} \cap {F_k} = \emptyset ,{F_j} \in \mathcal{M}$.
			
			Then Eq \ref{eq4.33} can be written as
			\begin{equation}
			{\pi ^*}\left( E \right) \geqslant {\pi ^*}\left( {E \cap \sum\limits_{j = 1}^n {{F_j}} } \right) + {\pi ^*}\left( {E\backslash A} \right)
			\label{eq4.36}
			\end{equation}
			By Remark \ref{rmk4.2}, we have 
				\begin{equation}
				\begin{split}
				{\pi ^*}\left( E \right) & \geqslant {\pi ^*}\left( {E \cap \left( {\sum\limits_{j = 1}^n {{F_j}} } \right)} \right) + {\pi ^*}\left( {E\backslash A} \right)\\
										 & = \sum\limits_{j = 1}^n {{\pi ^*}\left( {E \cap {F_j}} \right)}  + {\pi ^*}\left( {E\backslash A} \right)
				\end{split}
				\label{eq4.37}
				\end{equation}
				$\because \;n$ is any number in Remark \ref{rmk4.2} , $\therefore {\pi ^*}\left( {E \cap \sum\limits_{j = 1}^\infty  {{F_j}} } \right) = \sum\limits_{j = 1}^\infty  {{\pi ^*}\left( {E \cap {F_j}} \right)} $, by  $ {\pi ^*} $ is  subadditive
				\begin{equation}
				\begin{split}
				{\pi ^*}\left( E \right) & \geqslant {\pi ^*}\left( {E \cap \sum\limits_j {{F_j}} } \right) + {\pi ^*}\left( {E\backslash A} \right)\\
										 &  = \sum\limits_{j \geqslant 1} {{\pi ^*}\left( {E \cap {F_j}} \right)}  + {\pi ^*}\left( {E\backslash A} \right)\\
										 & \geqslant {\pi ^*}\left( {\bigcup\limits_{j \geqslant 1} {\left( {E \cap {F_j}} \right)} } \right) + {\pi ^*}\left( {E\backslash A} \right)\\
										 & =  \geqslant {\pi ^*}\left( {E \cap \left( {\bigcup\limits_{j \geqslant 1} {{F_j}} } \right)} \right) + {\pi ^*}\left( {E\backslash A} \right)\\
									   	 &  = {\pi ^*}\left( {E \cap A} \right) + {\pi ^*}\left( {E\backslash A} \right)
				\end{split}
				\label{eq4.38}
				\end{equation}
		\end{enumerate}
	So $ \mathcal{M} $ is a $ \sigma- $algebra.
	\end{enumerate}
\end{proof}

\begin{remark}
	$ \forall n, $ we have that
	\begin{equation}
	{\pi ^*}\left( {E \cap \sum\limits_{j = 1}^n {{F_j}} } \right) = \sum\limits_{j = 1}^n {{\pi ^*}\left( {E \cap {F_j}} \right)} 
	\label{eq4.39}
	\end{equation}
	where ${F_j}$ defined as Eq \ref{eq4.34}.
	\label{rmk4.2}
\end{remark}

\begin{proof}
	By induction
	\begin{enumerate}
		\item $ n = 1, $ Eq \ref{eq4.39} holds
		\item Support $ n $ holds then we will proof $  n + 1  $ holds.
	 ${F_k} \in \mathcal{M},{F_{n + 1}} \in \mathcal{M}$, we have that
	 \begin{equation}
	 \begin{split}
	 {\pi ^*}\left( {E \cap \sum\limits_{j = 1}^{n + 1} {{F_j}} } \right) & = {\pi ^*}\left( {\left( {E \cap \sum\limits_{j = 1}^{n + 1} {{F_j}} } \right) \cap {F_{n + 1}}} \right) + {\pi ^*}\left( {\left( {E \cap \sum\limits_{j = 1}^{n + 1} {{F_j}} } \right) \cap F_{_{n + 1}}^c} \right)\\
	                                                                      & = {\pi ^*}\left( {E \cap {F_{n + 1}}} \right) + \underbrace {{\pi ^*}\left( {E \cap \sum\limits_{j = 1}^n {{F_j}} } \right)}_{by\;\;assumption\;\;\; = \sum\limits_{j = 1}^n {{\pi ^*}\left( {E \cap {F_j}} \right)} }\\
	                                                                      & = \sum\limits_{j = 1}^{n + 1} {{\pi ^*}\left( {E \cap {F_j}} \right)} 
	 \end{split}
	 \label{eq4.40}
	 \end{equation}
	\end{enumerate}
\end{proof}

By Thm \ref{thm4.2} we have that $ \mathcal{M} \supseteq \mathcal{F}(a). $

{\large Step 3}

\begin{theorem}
	${\pi ^*}:\mathcal{M} \to {\mathbb{R}_ + } \cup \left\{ { + \infty } \right\}$ is $ \sigma- $ additive, then 
	\begin{equation}
	{\pi ^*}\left( A \right) = \nu \left( A \right),\;\;\forall A \in a
	\label{eq4.41}
	\end{equation}
	\label{thm4.3}
\end{theorem}

\begin{remark}
	Eq \ref{eq4.41} is also
	\begin{equation}
	{\pi ^*}{|_a} = v
	\label{eq4.42}
	\end{equation}
	Eq \ref{def4.2} holds because Thm \ref{thm4.2}, $ a \in \mathcal{M}. $
	\label{rmk4.3}
\end{remark}

\begin{proof}(Thm \ref{thm4.3})
	\begin{enumerate}
		\item ${\pi ^*}\left( A \right) = \nu \left( A \right),\;\forall A \in a$
		\begin{enumerate}
			\item check ${\pi ^*}\left( A \right) \leqslant \nu \left( A \right)$
			
			Let's $\underbrace A_{{E_1}},\underbrace \emptyset _{{E_2}},\underbrace \emptyset _{{E_3}},\underbrace  \cdots _{{E_j}}$
			\begin{equation}
			{\pi ^*}\left( A \right) = \mathop {\inf }\limits_{\left\{ {{E_i}} \right\},A} \sum\limits_i {\nu \left( {{E_i}} \right)}  \leqslant \sum\limits_i {\nu \left( {{E_i}} \right)}  = \nu \left( A \right)
			\label{eq4.43}
			\end{equation}
			\item check ${\pi ^*}\left( A \right) \geqslant \nu \left( A \right)$
			
			Let's take
			\begin{equation}
			\begin{split}
			{F_1} & = {E_1}\\
			{F_2} & = {E_2}\backslash {E_1}\\
			{F_3} & = {E_3}\backslash \left( {{E_1} \cup {E_2}} \right)\\
			      & \vdots \\
			{F_n} & = {E_n}\backslash \left( {{E_1} \cup {E_2} \cup  \cdots  \cup {E_{n - 1}}} \right)\\
			      & \vdots
			\end{split}
			\label{eq4.44}
			\end{equation}
			${F_j} \in a,\bigcup\limits_j {{F_j}}  = \bigcup\limits_j {{E_j}} ,{F_j} \cap {F_k} = \emptyset ,A \subseteq \bigcup\limits_{j \geqslant 1} {{F_j}} $, so $A = \sum\limits_j {{F_j} \cap A \in a} $. 
			
			Because $ \nu $  is $ \sigma- $additive we have that
			\begin{equation}
			\nu \left( A \right) = \sum\limits_{j \geqslant 1} {\nu \left( {{F_j} \cap A} \right)} 
			\label{eq4.45}
			\end{equation}
			
			$\because {F_j} \subseteq {E_j}$
			\begin{equation}
			\nu \left( A \right) = \sum\limits_{j \geqslant 1} {\nu \left( {{F_j} \cap A} \right)}  \leqslant \sum\limits_{j \geqslant 1} {\nu \left( {{E_j}} \right)} 
			\label{eq4.46}
			\end{equation}
			so
			\begin{equation}
			\nu \left( A \right) \leqslant \mathop {\inf }\limits_{\left\{ {{E_i}} \right\},A} \sum\limits_{j \geqslant 1} {\nu \left( {{E_j}} \right)}  = {\pi ^*}\left( A \right)
			\label{eq4.47}
			\end{equation}
		\end{enumerate}
	    Then, we can get
	    \begin{equation}
	    {\pi ^*}\left( A \right) = \nu \left( A \right),\;\forall A \in a
	    \label{eq4.48}
	    \end{equation}
		\item ${\pi ^*}{|_\mathcal{M}}$ is $ \sigma- $additive
		
		Suppose that $ A_{j} \in \mathcal{M}, {A_j} \cap {A_k} = \emptyset, $ we want to proof that
		\begin{equation}
		{\pi ^*}\left( {\sum {{A_j}} } \right) = \sum\limits_{j \geqslant 1} {{\pi ^*}\left( {{A_j}} \right)} 
		\label{eq4.49}
		\end{equation}
		\begin{enumerate}
			\item check ${\pi ^*}\left( {\sum {{A_j}} } \right) \leqslant \sum\limits_{j \geqslant 1} {{\pi ^*}\left( {{A_j}} \right)} $
			by ${\pi ^*}$ is an outer measure, ${\pi ^*}$ is subadditive
			\item check ${\pi ^*}\left( {\sum {{A_j}} } \right) \geqslant \sum\limits_{j \geqslant 1} {{\pi ^*}\left( {{A_j}} \right)} $
			
			by ${\pi ^*}$ is an outer measure, ${\pi ^*}$ is monotone
			\begin{equation}
			{\pi ^*}\left( {\sum\limits_{j \geqslant 1} {{A_j}} } \right) \geqslant {\pi ^*}\left( {\sum\limits_{j = 1}^n {{A_j}} } \right)
			\label{eq4.50}
			\end{equation}
			by Remark \ref{rmk4.2}, we have that
			\begin{equation}
			{\pi ^*}\left( {\sum\limits_{j = 1}^n {{A_j}} } \right) = \sum\limits_{j = 1}^n {{\pi ^*}\left( {{A_j}} \right)}, \ \ \forall n
			\label{eq4.51}
			\end{equation}
			so 
			\begin{equation}
			{\pi ^*}\left( {\sum\limits_{j \geqslant 1} {{A_j}} } \right) \geqslant \sum\limits_{j \geqslant 1} {{\pi ^*}\left( {{A_j}} \right)} 
			\label{eq4.52}
			\end{equation}
		\end{enumerate}
	\end{enumerate}
\end{proof}

{\large Step 4}

\begin{definition}
	$\Omega $ is $ \sigma $-finite$ (\mu_{1}) $  if ${E_j} \uparrow \Omega ,{\mu _1}\left( {{E_j}} \right) < \infty ,\;\;\forall j,\;{E_j} \in a$.
	\label{def4.4}
\end{definition}

\begin{theorem}[Uniqueness]
	Suppose that ${\mu _1},{\mu _2}:\mathcal{F}\left( a \right) \to {R_ + } \cup \left\{ { + \infty } \right\},\Omega $ is $ \sigma $-finite$ (\mu_{1}) $, if ${\mu _1}{|_a} = {\mu _2}{|_a}$, then 
	\begin{equation}
	{\mu _1} = {\mu _2}, \ \ on\ \ \mathcal{F}(a)
	\label{eq4.53}
	\end{equation}
	\label{thm4.4}
\end{theorem}

\begin{definition}
	$ \Omega $, $ \mathcal{G} \subseteq \mathcal{P}\left(\Omega\right), \mathcal{G} $ is a monotone class if
	\begin{enumerate}
		\item  
		\begin{equation}
		{A_j} \in \mathcal{G},j \geqslant 1,{A_j} \subseteq {A_{j + 1}} \Rightarrow A = \bigcup\limits_{j \geqslant 1} {{A_j}}  = \mathop {\lim }\limits_{j \to \infty } {A_j} \in \mathcal{G}
		\label{eq4.54}
		\end{equation}
		\item 
		\begin{equation}
		{B_j} \in \mathcal{G},j \geqslant 1,{B_j} \supseteq {B_{j + 1}} \Rightarrow B = \bigcap\limits_{j \geqslant 1} {{B_j}}  = \mathop {\lim }\limits_{j \to \infty } {B_j} \in \mathcal{G}
		\label{eq4.55}
		\end{equation}
	\end{enumerate}
	\label{def4.5}
\end{definition}

\begin{theorem}
	${\mathcal{G}_\alpha }$ is a monotone class, $ \alpha \in I $, then the followings hold
	\begin{enumerate}
		\item $\bigcap\limits_{\alpha  \in I} {{\mathcal{G}_\alpha }} $ is a monotone class
		\item $c \subseteq \mathcal{P}\left( \Omega  \right) \Rightarrow \mathcal{G}\left( c \right) = \bigcap\limits_{\alpha  \in I} {{\mathcal{G}_\alpha }} $ , i.e. monotone classes generated by class $ c $
	\end{enumerate}
	\label{thm4.5}
\end{theorem}

\begin{lemma}
	$ a \subseteq \mathcal{P}\left(\Omega\right) $ is an algebra, $\mu \left( a \right)$ is monotone class generated by algebra $ a $, $ \mathcal{F}\left(a\right) $ is a $ \sigma- $algebra generated by algebra $ a $, then
	\begin{equation}
	\mu \left( a \right) = \mathcal{F}\left( a \right)
	\label{eq4.56}
	\end{equation}
	\label{lem4.1}
\end{lemma}
\begin{proof}
	It will proof in the next lecture.
\end{proof}

\begin{proof}(Thm \ref{thm4.4})
	${\mu _1},{\mu _2}:\mathcal{F}\left( a \right) \to {\mathbb{R}_ + } \cup \left\{ { + \infty } \right\}$, ${\mu _1}\left( A \right) = {\mu _2}\left( A \right),\forall A \in a$, $\Omega \;\sigma $-finite, $\Omega  = \bigcup\limits_{j \geqslant 1} {{E_j}} ,{E_j} \in a,{\mu _j}\left( {{E_j}} \right) < \infty $, then ${\mu _1} = {\mu _2}$ on $ \mathcal{F}(a). $
	
	Fix ${E_n}$, we denote that
	\begin{equation}
	{\mathcal{B}_n} = \left\{ {E \in \mathcal{F}\left( a \right),{\mu _1}\left( {E \cap {E_n}} \right) = {\mu _2}\left( {E \cap {E_n}} \right)} \right\}
	\label{eq4.57}
	\end{equation}
	We claim that
	\begin{enumerate}
		\item $ {\mathcal{B}_n} \supseteq a $
		\item $ {\mathcal{B}_n} $ is a monotone class 
	\end{enumerate}
    We proof $ {\mathcal{B}_n} $ is a monotone class.
   \begin{enumerate}
   	\item $\forall {A_j} \in {\mathcal{B}_n},{A_j} \uparrow A = \bigcup\limits_{j \geqslant 1} {{A_j}} $, then 
   	\begin{equation}
   	{\mu _1}\left( {{A_j} \cap {E_n}} \right) = {\mu _2}\left( {{A_j} \cap {E_n}} \right)
   	\label{eq4.58}
   	\end{equation}
   	By Remark \ref{lem3.1}
   	\begin{equation}
   	{\mu _1}\left( {{A_j} \cap {E_n}} \right) \to {\mu _1}\left( {A \cap {E_n}} \right),{\mu _2}\left( {{A_j} \cap {E_n}} \right) \to {\mu _2}\left( {A \cap {E_n}} \right)
   	\label{eq4.59}
   	\end{equation}
   	\item $\forall {B_j} \in {\mathcal{B}_n},{B_j} \downarrow B = \bigcap\limits_{j \geqslant 1} {{B_j}} $, then 
   	\begin{equation}
   	{\mu _1}\left( {{B_j} \cap {E_n}} \right) = {\mu _2}\left( {{B_j} \cap {E_n}} \right)
   	\label{eq4.60}
   	\end{equation}
   	By Remark \ref{lem3.1}
   	\begin{equation}
   	{\mu _1}\left( {{B_j} \cap {E_n}} \right) \to {\mu _1}\left( {B \cap {E_n}} \right),{\mu _2}\left( {{B_j} \cap {E_n}} \right) \to {\mu _2}\left( {B \cap {E_n}} \right)
   	\label{eq4.61}
   	\end{equation}
   \end{enumerate} 
   So we can get that 
   \begin{equation}
    {\mathcal{B}_n} \supseteq \mathcal{M}\left( a \right)
    \label{eq4.62}
   \end{equation}
   where 
   $ \mathcal{M}\left( a \right) $ is a monotone class generated by $ a $. Then by Lemma \ref{lem4.1} 
   \begin{equation}
   \mathcal{M}\left( a \right) = \mathcal{F}\left( a \right)
   \label{eq4.63}
   \end{equation}
   And by Eq \ref{eq4.57},
   \begin{equation}
   {\mathcal{B}_n}\left( a \right) \subseteq \mathcal{F}\left( a \right)
   \label{eq4.64}
   \end{equation}
   so
   \begin{equation}
   {\mathcal{B}_n}\left( a \right) = \mathcal{F}\left( a \right)
   \label{eq4.65}
   \end{equation}
   Finally, 
   ${\mu _1}\left( A \right) = {\mu _2}\left( A \right),\forall A \in \mathcal{F}\left( a \right)$, by $ {\mathcal{B}_n} =  \mathcal{F}(a),  $  then  $ A \in {\mathcal{B}_n}$. ${B_j} \uparrow \Omega $, apply Lemma \ref{lem3.1} again, we have 
   \begin{equation}
   {\mu _1}\left( A \right) = {\mu _2}\left( A \right)
   \end{equation}
\end{proof}
